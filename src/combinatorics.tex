\documentclass{article}
\usepackage{booktabs}
\usepackage{amsmath}
\usepackage{amsfonts}
\usepackage{tikz}
\usepackage{xcolor}
\usepackage{cancel}
\usepackage{array}

\title{Combinatorics}

\begin{document}
    \maketitle

    \section{Basis}

    The best way to think about combinations and permutations is to think about all permutations of $n$ things: \begin{equation}
        n! = P(n, n)
    \end{equation}

    Then we think about discounting orderings that are the same: \begin{equation}
        \frac{n!}{n_1! n_2! \cdots n_m!} = {n \choose n_1, n_2, ..., n_m}, \quad \text{where } \sum_{k = 1}^m{n_k} = n
    \end{equation}

    This is referred to as \textbf{permutations of multisets}. These represent items in the set that are considered the same.

    From this, we can derive all other types of combinations and permutations.

    For standard \textbf{permutations}: \begin{equation}
        P(n, r) = {n \choose \underbrace{1, 1, ..., 1}_{r \text{ times}}, n - r}
    \end{equation}

    In fact, since the sum of the $n_k$'s must be $n$, there are always a number of implicit 1's to add up to $n$, e.g.: \begin{equation}
        {n \choose n_1, ..., n_m} = {n \choose n_1, ..., n_m, \underbrace{1, ..., 1}_{n - k \text{ times}}}, \quad \text{where } k = \sum_{i = 1}^m{n_k} < n
    \end{equation}

    The one exception is when there is only one term on the bottom, then there is an implicit $n - r$ term as well: \begin{equation}
        {n \choose r} = {n \choose r, n - r}
    \end{equation}

    These are standard \textbf{combinations}.
    
    \section{Alternate Basis}

    Another way to think about combinations and permutations is to start with combinations: \begin{equation}
        C(n, r) = {n \choose r} = \frac{n!}{r!(n - r)!}
    \end{equation}

    Permutations can then be thought of as reordering $n$ things: \begin{equation}
        P(n, n) = n!
    \end{equation}
        
    To permute a subgroup, first \textit{choose} the subgroup, then permute: \begin{equation}
        P(n, r) = \color{purple}{n \choose r}r!\color{black} = \frac{n!}{\cancel{r!}(n - r)!}\cancel{r!} = \frac{n!}{(n - r)!}, \quad r \leq n
    \end{equation}

    \section{Identities}
    
    \begin{equation}
        {n \choose 0} = P(n, 0) = 1
    \end{equation}

    \begin{equation}
        {0 \choose r} = P(0, r) = 0
    \end{equation}
    
    \begin{equation}
        {n \choose 1} = P(n, 1) = n
    \end{equation}
    
    \begin{equation}
        {n \choose r} = {n \choose n - r}
    \end{equation}

    \begin{equation}
        {n \choose r} = {n - 1 \choose r} + {n - 1 \choose r - 1}
    \end{equation}

    \section{Notes}

    Permutations of multisets (repeats in input): \begin{equation}
        P_M(n, x, y, z) = {n \choose x, y, z} = \frac{n!}{x!y!z!}
    \end{equation}

    Circular permutations: \begin{equation}
        P_C(n) = (n - 1)!
    \end{equation}

    \section{Identities}

    $$P(n, 0) = 1$$

    $$P(n, 1) = n$$

    $$P(n, n) = n!$$
    
    $$P(n, n - 1) = P(n, n)$$
    
    $${n \choose n - r} = {n \choose r}$$

    \section{Repeats}

    $k$ consecutive items: \begin{equation}
        P(n - k + 1, r)
    \end{equation}

    Count $n$ minus $k$ consecutive items, plus 1 representing the items as a group.

    $k$ items can't be together: \begin{equation}
        P(n - k, r) \cdot P(n - k + 1, k)
    \end{equation}

    Count all $n - k$ items without restriction. Then place $k$ items in $n - k$ spaces after of each item, and 1 space before first item. Then the items

    \begin{table}[h]
        \caption{Combinatorial Operations}
        \centering
        \begin{tabular}{cl}
            \toprule
            Operation & Meaning \\
            \midrule
            $+ n$ & Add $n$ choices to an event \\
            $- n$ & Remove $n$ choices from an event \\
            $\times n$ & Add an event with $n$ choices \\
            $\div n$ & Remove an event with $n$ choices \\
            \bottomrule
        \end{tabular}
    \end{table}

    \section{Integer Solutions to Linear Expressions}

    $$x_1 + x_2 + x_3 + \cdots + x_n = r$$
    
    $${n + r - 1 \choose r}$$

    $$x_1 + x_2 + x_3 + \cdots + x_n \leq r$$
    
    $${n + r \choose r}$$

    $$x_1 + x_2 + x_3 + \cdots + x_n < r$$
    
    $${n + r - 1 \choose r - 1}$$
\end{document}
